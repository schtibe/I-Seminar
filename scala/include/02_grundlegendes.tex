\chapter{Grundlegendes}

\section{Hallo Welt}
\label{sec:helloWorld}

Als erstes Beispiel soll, wie üblich, ein Hallo-Welt-Programm dienen:

\lstset{float=ht,language=scala,caption={Hallo Welt},label=lst:helloWorld}
\lstinputlisting{src/HelloWorld.scala}

Die Definition der Main-Methode ist dabei analog zur Main-Methode in
einer Java-Applikation.


\section{Kompilieren und Ausführen}

Scala bietet drei Möglichkeiten zum Ausführen von Code. Einerseits kann Scala
für umfangreiche Applikationen kompiliert werden. Dies wird durch den
Aufruf des Scala-Compilers \emph{scalac} erreicht. Dabei werden Dateien
mit der Endung \emph{.class} erstellt. Ausgeführt kann eine Applikation
schliesslich durch den Aufruf von \emph{scala}, bei dem der Klassenname 
ohne Dateiendung mitgegeben wird, wie bei Java:

\begin{lstlisting}[float=ht,language=bash,caption=Kompilieren und ausführen von Scala-Code,label=lst:execute]
scalac HelloWorld.scala
scala HelloWorld
=> Hello World!
\end{lstlisting}

Eine andere Möglichkeit besteht darin, Scripts zu erstellen, welche dann
während der Laufzeit interpretiert und ausgeführt werden. Dazu kann
der Name der Datei direkt an \emph{scala} übergeben werden. Die Definition
einer Main-Methode ist hierbei nicht benötigt.\\

Die dritte Möglichkeit ist, interaktiv Scala-Code in den Interpreter 
einzugeben. Der Interpreter wird gestartet, wenn \emph{scala} ohne
Argumente aufgerufen wird.

\section{Var und Val}

Scala kennt zwei Keywords, um Variablen zu definieren: \emph{val} und 
\emph{var}. Der Unterschied besteht darin, dass eine Variable, die mit
\emph{val} definiert wurde, nicht mehr verändert werden kann. \emph{val}
ist also ähnlich wie eine \emph{final}-Variable in Java.

\section{Datentypen}
\label{sec:datatypes}

\subsection{Datentypen sind Objekte}

Scala ist eine pure Objekt-Orientierte Sprache. Dies bedeutet, dass alles
ein Objekt ist, selbst Funktionen (siehe \ref{sec:functionsObjects}). Im
Gegensatz zu Java gibt es also auch keine Primitiven Datentypen.\\

Eine Spezialität ist, dass auch arithmetische Operationen eigentlich
Methodenaufrufe darstellen. So wird folgender Aufruf

\begin{lstlisting}float=ht,language=scala,caption=Arithmetische Operationen,label=lst:arithmetic]
2 + 3 * 4
\end{lstlisting}

umgewandelt zu
\begin{lstlisting}float=ht,language=scala,caption=Arithmetische Operationen Konvertiert,label=lst:arithmeticConverted]
(2).+((3).*(4))
\end{lstlisting}

Interessant anzumerken ist hierbei noch, dass arithmetische Zeichen
als Methodennamen erlaubt sind.

\subsection{Literale}
Nur kurz da nicht allzusehr interessant!

- integer
- floats
- characters
- string
- symbole
- boolean


\subsection{Spezielle Datentypen}

- Unit
- Any
- lists
- tuples
- sets and maps
