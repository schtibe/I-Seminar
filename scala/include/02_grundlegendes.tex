\chapter{Grundlegendes}

\section{Hallo Welt}

\section{Kompilieren und Ausführen}

\section{Datentypen}
\label{sec:datatypes}

\subsection{Datentypen sind Objekte}

Scala ist eine pure Objekt-Orientierte Sprache. Dies bedeutet, dass alles
ein Objekt ist, selbst Funktionen (siehe \ref{sec:functionsObjects}). Im
Gegensatz zu Java gibt es also auch keine Primitiven Datentypen.\\

Eine Spezialität ist, dass auch arithmetische Operationen eigentlich
Methodenaufrufe darstellen. So wird folgender Aufruf

\begin{lstlisting}float=ht,language=scala,caption=Arithmetische Operationen]
2 + 3 * 4
\end{lstlisting}

umgewandelt zu
\begin{lstlisting}float=ht,language=scala,caption=Arithmetische Operationen]
(2).+((3).*(4))
\end{lstlisting}

Interessant anzumerken ist hierbei noch, dass arithmetische Zeichen
als Methodennamen erlaubt sind.

\subsection{Literale}
Nur kurz da nicht allzusehr interessant!

- integer
- floats
- characters
- string
- symbole
- boolean


\subsection{Spezielle Datentypen}

- Unit
- Any
- lists
- tuples
- sets and maps
