\chapter{Konzepte}

\section{Variablen}

Scala kennt zwei Keywords, um Variablen zu definieren: \emph{val} und 
\emph{var}. Der Unterschied besteht darin, dass eine Variable, die mit
\emph{val} definiert wurde, nicht mehr verändert werden kann. \emph{val}
ist also ähnlich wie eine \emph{final}-Variable in Java.

\section{Objektorientierung}

\subsection{Singleton-Objekte}
\label{sec:singleton}

Da Scala eine objektorientierte Programmiersprache ist, können nebst
den eingebauten Datentypen (\ref{sec:datatypes}) auch Klassen oder
Objekte definiert werden. Klassen haben die gleiche Funktionalität
wie in Java, mehr dazu unter \ref{sec:classes}.  Anders als in Java
ist die Möglichkeit, direkt Objekte zu definieren.  Dies kann mit dem
\emph{object}-Keyword erreicht werden, und entspricht dem Konzept des
allgemein bekannten Singleton-Pattern (es kann also nur eine Instanz davon
geben). Diese einmalige Instanz eines solchen Objekts wird beim ersten
Gebrauch automatisch kreiert. Ein wichtiger Unterschied zu Klassen ist,
dass ein solches Singleton-Objekt keine Parameter entgegen nehmen kann,
da es nicht manuell instanziert werden kann.

Ein Keyword wie \emph{static} in Java gibt es nicht. Anstatt
Methoden statisch zu implementieren, ist es in Scala üblich, sie in
Singleton-Objekten zu definieren.\\

Ein Beispiel eines einfachen Singleton-Objekts bietet das
Hallo-Welt-Beispiel im Kapitel \ref{sec:helloWorld}.

\subsection{Klassen}
\label{sec:classes}

\subsubsection{Deklaration und Konstruktoren}

Klassen werden in einer Syntax deklariert, die der von Java recht
ähnlich ist. Der Klassen-Body kann dabei verwendet werden,
um Operationen bei der Instanzierung der Klasse auszuführen. In Java
wird dazu eine Konstruktor-Methode benötigt, in Scala hingegen können
diese Operationen einfach unter den Klassennamen geschrieben werden. \\

Ein Beispiel einer einfachen Klasse mit Konstruktor in Java
\lstset{float=ht,language=java,caption={Java-Konstruktor},label=lst:javaConstructor}
\lstinputlisting{src/Constructing.java}

würde in Scala wie folgt aussehen:
\lstset{float=ht,language=scala,caption={Scala-Konstruktor},label=lst:scalaConstructor}
\lstinputlisting{src/Constructing.scala}

Dies wird der \emph{primary constructor} benannt. Dem Programmierer steht es 
frei, noch weitere Konstruktoren zu definieren.


\subsubsection{Instanzierung}

Die Instanzierung von Klassen erfolgt analog zur Java-Syntax. So würde das 
Konstruktor-Beispiel im vorigen Kapitel (\ref{lst:scalaConstructor}) wie 
folgt instanziert werden:

\lstset{float=ht,language=scala,caption={Instanzieren in Scala},label=lst:instancing}
\lstinputlisting{src/TestConstructing.scala}

Es gibt noch den Spezialfall Case-Klassen (\ref{sec:patternMatching}),
bei denen das \emph{new}-Keyword weggelassen werden darf.

\subsubsection{Komposition und Vererbung}

Scala bietet alle nötigen Instrumente, um Klassen zu vererben und Subtypen
zu erstellen. Ebenfalls erstellt werden können abstrakte Klassen, finale
Klassen und Variablen. Ebenso möglich ist es, die Sichtbarkeit von 
Klassenmember mit \emph{private} oder \emph{protected} einzuschränken. 
Kenner von Java finden sich mit den entsprechenden Keywords 
schnell zurecht.\\

Ein Keyword, welches in Scala zusätzlich anzutreffen ist, ist 
\emph{override}. Falls ein Klassen-Member überschrieben werden soll,
muss dieses Keyword vor die Definition gestellt werden. Dies macht
ein versehentliches Überschreiben von Klassen-Member unmöglich.\\

Ein Umfangreiches Beispiel von Klassenvererbung und Komposition bietet
Beispiel wird im Kapitel \ref{sec:bspClasses} gezeigt.

\subsection{Traits}

Ein Konzept, das beim Vergleich mit Java zu fehlen scheint, sind 
Interfaces. Scala bietet hierfür jedoch Traits, welche
den Interfaces nicht unähnlich sind. Traits können wie Klassen
vererbt werden, was als \glqq mixin\grqq \footnote{Vom Englischen \emph{to mix in}}
bezeichnet wird. Der hauptsächliche Unterschied zu Interfaces besteht 
darin, dass Traits bereits Implementation von Methoden enthalten
können. Jedoch können sie keine Konstruktor-Parameter enthalten,
wie es bei abstrakten Klassen möglich ist.\\

Es dürfen mehrere Traits in eine Klasse \glqq gemischt\grqq werden. Dies
ähnelt der Mehrfachvererbung von z.B. c++. Im Gegensatz dazu können
jedoch mit Traits Probleme
\footnote{Das Diamond-Problem 
\href{http://de.wikipedia.org/wiki/Diamond-Problem}
{http://de.wikipedia.org/wiki/Diamond-Problem}} vermieden werden, die 
normalerweise bei Mehrfachvererbung auftauchen. Dies wird erreicht,
indem mit Traits keine Hierarchie dargestellt werden kann. Dieses
Konzept wird auch horizontale Vererbung genannt.

\section{Funktionale Konzepte}

\subsection{Funktionen als Objekte und anonyme Funktionen}
\label{sec:functionsObjects}

Eines der Konzepte in Scala, welches durch die funktionale Programmierung
inspiriert wurde, ist die Eigenschaft, dass Funktionen auch Objekte
sind. Dies macht es möglich, Funktionen in Variablen zu speichern, sie
als Argument an Funktionen zu übergeben oder als Return-Werte zurück zu
geben. Ein Anwendungsbereich für eine solche Funktion könnte z.B. ein
Callback sein, der auf einem Event registriert wird.\\

Eine solche Funktion kann einerseits eine benannte Funktion sein, auch
eine die innerhalb einer Klasse oder einem Objekt definiert ist.  Falls
sie jedoch nur dazu definiert wird, um z.B. an eine Funktion übergeben
zu werden, ist eine Benennung der Funktion überflüssig. Aus diesem Grund
kann eine Funktion auch \emph{anonym} erstellt und verwendet werden.\\

Dies wird wie folgt erreicht: 

\begin{lstlisting}[float=ht,language=scala,caption=Anonyme Funktion,label=lst:anonymousFunction]
(x: Int) => x + 1
\end{lstlisting}

Das Beispiel im Kapitel \ref{sec:bspAnonymous} zeigt eine Verwendung einer solchen 
anonymen Funktion, welche einem Timer übergeben wird und jede Sekunde
aufgerufen wird. \\

\subsection{Pattern Matching}
\label{sec:patternMatching}

Ein anderes Konzept der Funktionalen Programmierung ist das des Pattern
Matching. Hierbei können Muster angegeben werden, nach denen die
Eingabeparameter einer Funktion geprüft werden. Zusätzlich zu den
Mustern werden verschiedene Funktionen definiert. Es wird dann diejenige
Funktion ausgeführt, bei der das Muster auf die Eingabe zutrifft. Eine
Funktion mit Pattern Matching bietet also je nach Eingabe unterschiedliche
Definitionen. \\

In Scala kann Pattern Matching unter anderem auf sogenannte \emph{Case
Classes} angewendet werden. So können Funktionen erstellt werden, die
sich je nach Eingabe, sei es Typen oder Werte, anders verhalten.  Möchte
man das gleiche in Java erreichen, müsste man ein switch-Statement
oder sogar eine \glqq if-elseif-Verzweigung\grqq konstruieren, welches
alle möglichen Eingabewerte (unter Umständen mit \emph{instanceof})
überprüft oder gar das Visitor-Pattern implementieren.

Case-Classes können einfach erstellt werden, indem lediglich das Keyword
\emph{case} vor die Klassendeklaration gestellt wird. Diese werden
dann so auf die Verwendung in Pattern Matches \glqq vorbereitet\grqq .\\

Um Pattern Matching etwas näher zu bringen, erstmal ein Beispiel einer
Funktion, welche verschiedene Eingabewerte \glqq matcht\grqq und dementsprechend
ein verschiedenes Verhalten an den Tag legt (dieses Beispiel kommt der 
Einfachheit halber ohne Case-Classes aus)

\begin{lstlisting}[float=ht,language=scala,caption=Pattern Matching auf Literale \cite{odersky},label=lst:patternMatching]
def describe (x: Any) = x match {
	case 5       => "five"
	case true    => "truth"
	case "hello" => "hi!"
	case Nil     => "the empty list"
	case _       => "something else" // Default
}
\end{lstlisting}

Ein wichtiger Vorteil gegenüber einer Implementation in Java mit
einem Switch-Statement und \emph{instanceof} sind sogenannte 
\emph{Constructor Patterns}. Hierbei kann nicht nur das Objekt selber
verglichen werden sondern auch dessen Inhalt. \\

Pattern Matching kommt oft dann zum Zug, wenn es um Baum-ähnliche
Datenstrukturen geht. Das Beispiel im Kapitel \ref{sec:caseClasses} zeigt ein solches
Beispiel zusammen mit Case Classes.

\subsection{Closures}

Closures bilden ein Konzept in Funktionen, welches beim Aufruf der 
Funktion den Kontext, der bei der Erstellung der Funktion vorherrschte,
wiederherstellt. \\

Ein Beispiel soll dies verdeutlichen:

\lstset{float=ht,language=scala,caption={Eine Closure \cite{odersky}},label=lst:closures}
\lstinputlisting{src/Closures.scala}

Ausgeführt gibt dieses Beispiel den Wert 11 aus.

Obwohl die Variable \emph{more} im Kontext der Funktion nicht gebunden
wurde, funktioniert dieser Code weil im Kontext der Erstellung eine
Variable \emph{more} existierte. Wird diese Variable nach der
Erstellung der Closure verändert, wird dies von der Closure bemerkt und 
der veränderte Wert verwendet. Umgekehrt wird eine Veränderung der
Variable innerhalb der Funktion auch nach aussen sichtbar.

\subsection{Currying}

Scala unterstützt auch Currying. Hierbei wird kann eine Funktion,
die \emph{n} Argumente erwartet, mit nur einem Argument aufgerufen werden. 
Daraus resultiert eine neue Funktion die nur noch \emph{n - 1} Argumente
erwartet und das bereits gegebene Argument fest enthält.\\

Folgend ein Beispiel, welches die Verwendung einer solchen Funktion
verdeutlicht: 

\lstset{float=ht,language=scala,caption={Eine Closure \cite{odersky}},label=lst:closures}
\lstinputlisting{src/Currying.scala}

Dieses Beispiel gibt das korrekte Resultat 5 aus.


\section{Typeninferenz}

Scala ist statisch typisiert. Dies bedeutet, dass der Datentyp von
Variablen bereits zur Kompilierzeit festgelegt wird. Dies kann in Scala
durch den Programmierer explizit gemacht werden. Zusätzlich bietet Scala
jedoch noch die sogenannte \emph{Typinferenz}: es ist oftmals möglich,
die explizite Angabe von Datentypen wegzulassen da der Compiler in der
Lage ist, manche Datentypen aus dem Kontext herzuleiten. Als Beispiel
soll eine einfache Deklaration einer Variable dienen, bei der der Datentyp
(String) vom Compiler abgeleitet werden kann:

\begin{lstlisting}[float=ht,language=scala,caption=Automatisch hergeleiteter Datentyp,label=lst:typeinference]
var foo = "bar"
\end{lstlisting}


\section{XML}

Scala unterstützt XML nativ. So kann XML einfach als Literal in den
Scala-Code geschrieben werden (ohne Anführungszeichen). Innerhalb
dieser XML-Literale kann wiederum Scala-Code geschrieben werden, der
evaluiert wird.  Weiter kann unter anderem Pattern Matching angewendet
werden, um durch XML zu traversieren. \\

Ein kleines Beispiel zur Verwendung von nativem XML in Scala:
\begin{lstlisting}[float=ht,language=scala,caption={XML mit Code, der evaluiert wird},label=lst:xml]
var foo = <a> { 3 + 5 } </a>
\end{lstlisting}


