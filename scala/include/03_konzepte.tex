\chapter{Konzepte}

\section{Objektorientierung}

- Alles sind objekte
- "object" kreiert ein Singleton Objekt (klasse *und* Instanz)
- Static gibts nicht -> Singleton

- Getter methoden für Klassenvariablen

(- override)


\section{Funktionale Konzepte}

\subsection{Funktionen als Objekte und anonyme Funktionen}

Eines der grundlegenden Konzepte aus der Funktionalen Programmierung ist,
dass Funktionen auch Objekte sind. Dies macht es möglich, Funktionen
in Variablen zu speichern, sie als Argument an Funktionen zu übergeben
oder als Return-Werte zurück zu geben. Ein Anwendungsbereich für eine
solche Funktion könnte z.B. ein Callback sein, der auf einem Event
registriert wird.\\

Eine solche Funktion kann einerseits eine benannte Funktion sein,
auch eine die innerhalb einer Klasse oder einem Objekt definiert ist.
Falls sie jedoch nur zu diesem einen Zweck definiert wird, ist eine 
Benennung der Funktion überflüssig. Aus diesem Grund kann eine Funktion
auch \emph{anonym} erstellt und verwendet werden.\\

Beispiel \ref{lst:timerAnonymous} zeigt eine Verwendung einer solchen 
anonymen Funktion, welche einem Timer übergeben wird und jede Sekunde
aufgerufen wird. \\


- Funktionen sind auch objekte => return vals, arguments

Beispiel mit dem Timer!

\subsection{Pattern Matching}

\subsection{Closures}

\subsection{Currying}

\section{Typensystem}

\subsection{Typinferenz}

\section{XML}

\section{Threading}
