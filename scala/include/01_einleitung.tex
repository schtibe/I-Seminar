\chapter{Einleitung}

\section{Geschichte}



\section{Funktionale Programmierung}

Funktionale Programmierung ist ein Programmierparadigma, bei welchem
ein Programm ausschliesslich aus Funktionen besteht. Im Gegensatz
zu imperativen Programmen, die aus Rechenanweisungen bestehen, sind
Funktionale Programme eine Menge von Funktions-Definitionen, die man
mathematisch als partielle Abbildungen von Eingabedaten auf Ausgabedaten
auffassen kann \cite{wikipediaFunktional}. Funktionale Programmierung
entspringt dem Lamda-Kalkül und entspricht generell der mathematischer
Auffassung von Funktionen. Die Konzepte von Funktionalen
Programmiersprachen bieten somit eine andere Herangehensweise an Probleme.\\

Scala ist zwar einerseits eine reine Objektorientierte Programmiersprache,
sie implementiert jedoch einige Konzepte aus der Welt der Funktionalen
Programmierung, namentlich \emph{Anonyme Funktionen}, \emph{Currying},
\emph{Pattern Matching} sowie \emph{Closures}.

