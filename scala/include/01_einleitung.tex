\chapter{Einleitung}

\section{Geschichte}

Scala ist eine Programmiersprache, welche funktionale sowie
Objektorientierte Programmierung unter einem Hut vereint. Sie ist statisch
Typisiert und kann sowohl als Script-Sprache verwendet als auch kompiliert
werden. Bezeichnend ist, dass Scala-Code oft viel kürzer ist als ein
Java-Äquivalent.\\

Die Konzeptionierung von Scala wurde 2001 in der École Polytechnique
Fédérale de Lausanne von Martin Odersky gestartet. Odersky hat
vorher die aktuelle Version von javac sowie Generic Java 
\footnote{Generic Java war eine Sprache, die Java um Generics erweiterte
bevor es direkt in Java implementiert wurde}
entwickelt. Er war zudem seit 1999 damit beschäftigt, die Verbindung
funktionaler und objektorienterter Programmierung zu erforschen, woraus
schliesslich das Scala-Projekt entstand. \\

2004 wurde schliesslich Scala auf der Java-Plattform herausgebracht. Eine
zweite Version erschien 2006.

\section{Funktionale Programmierung}

Funktionale Programmierung ist ein Programmierparadigma, bei welchem
ein Programm ausschliesslich aus Funktionen besteht. Im Gegensatz
zu imperativen Programmen, die aus Rechenanweisungen bestehen, sind
Funktionale Programme eine Menge von Funktions-Definitionen, die man
mathematisch als partielle Abbildungen von Eingabedaten auf Ausgabedaten
auffassen kann \cite{wikipediaFunktional}. Funktionale Programmierung
entspringt dem Lamda-Kalkül und entspricht generell der mathematischer
Auffassung von Funktionen. Die Konzepte von Funktionalen
Programmiersprachen bieten somit eine andere Herangehensweise an Probleme.\\

Scala ist zwar eine reine Objektorientierte Programmiersprache, sie
implementiert jedoch einige Konzepte aus der Welt der Funktionalen
Programmierung, namentlich \emph{Anonyme Funktionen}, \emph{Currying},
\emph{Pattern Matching} sowie \emph{Closures}. Dies ermöglicht es
dem Programmierer, manche Probleme auf eine elegantere Art und Weise
zu lösen, als es mit einer rein objektorientierten Sprache wie Java
möglich wäre.

