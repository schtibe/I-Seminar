\chapter{Beispiele}


\section{Klassen und Vererbung}
\label{sec:bspClasses}

Dieses Beispiel zeigt die Verwendung von Klassen und Vererbung in Scala
(alle Klassen in Scala erben von der Klasse \emph{Any}). Sie bildet
folgende Klassenkonstellation ab:


\begin{figure}[H]
	\centering
	\begin{emp}[classdiag](20, 20)
		Class.Any("Any")()();

		AbstractClass.Shape("Shape")("+id: String", "x: Int", "y: Int")
			("+move(x: Int, y: Int)", "+draw()", "toString(): String", "getLabel(): String", "getShapeInfo(): String");
		Class.Ellipse("Ellipse")("radius: Int", "stretch: Int")();
		Class.Rectangle("Rectangle")("w: Int", "h: Int")();
		Class.Circle("Circle")("{r = s}")();
		Class.Square("Square")("{w = h}")();

		Shape.n = Any.s + (0, -20);
		Ellipse.ne = Shape.s + (-10, -20);
		Rectangle.nw = Shape.s + (10, -20);
		Circle.n = Ellipse.s + (0, -20);
		Square.n = Rectangle.s + (0, -20);

		drawObjects(Any, Shape, Ellipse, Rectangle, Circle, Square);
		link(inheritance)(Ellipse.n -- Shape.s);
		link(inheritance)(Rectangle.n -- Shape.s);
		link(inheritance)(Circle.n -- Ellipse.s);
		link(inheritance)(Square.n -- Rectangle.s);
		link(inheritance)(Shape.n -- Any.s);

	\end{emp}
	\caption{Beispiel einer Klassenhierarchie}
	\label{fig:covariance}
\end{figure}

Eine Implementation in Scala könnte wie folgt aussehen: 

\lstset{float=ht,language=scala,caption={Eine Klassenkonstellation},label=lst:classConstellation}
\lstinputlisting{src/shapes/Shapes.scala}

Auffällig ist hier, dass die Klammern bei den Funktionsaufrüfen \emph{this.getLabel} 
und \emph{this.getShapeInfo} weggelassen wurde. Dies ist erlaubt, solange eine Funktion
keine Argumente erwartet. \emph{Shape} hat ausserdem einen Primary Constructor, welcher prüft,
dass die Positionsargumente positiv sind.

\section{Timer mit anonymer Funktion}
\label{sec:bspAnonymous}

Dieses Beispiel ruft jede Sekunde eine Funktion auf, die als
anonyme Funktion übergeben wurde.
Quelle: \cite{scalaTutorial}

\lstset{float=ht,language=scala,caption={Timer mit anonymer Funktion},label=lst:timerAnonymous}
\lstinputlisting{src/anonymousTimer.scala}

\section{Case Classes und Pattern Matching}
\label{sec:caseClasses}

Dieses Beispiel (Quelle \cite{odersky}) bietet eine Datenstruktur
bestehend aus Case Classes, um arithmetische Ausdrücke zu
repräsentieren. Zusätzlich definiert es eine Funktion mit Pattern
Matching, welche diese Ausdrücke vereinfachen kann.  Namentlich wird
folgendes gemacht:

\begin{itemize}
\item Doppelte Negation wird entfernt: $--15$ wird zu $-15$
\item Addition mit 0 wird entfernt: $3 + 0$ wird zu $3$
\item Multiplikation mit 1 wird entfernt: $42 * 1$ wird zu $42$
\end{itemize}

\lstset{float=ht,language=scala,caption={Vereinfachen von arithmetischen Ausdrücken mit Pattern Matching},label=lst:patternMatching}
\lstinputlisting{src/ExpressionSimplifier/ExpressionSimplifier.scala}
