\chapter{Schlussbericht}

\section{Fazit}
Aus den Drei Motion-Capture Systemen hat sich definitiv das optische Motion Capturing System herauskristallisiert. Durch die einfache Integration der Schauspieler und Bewegungsfreiheit, werden Abdeckungssprobleme und somit die Nacharbeitung am Computer gerne in Kauf genommen. \\
Natürlich ist Motion Capturing nicht für jedes Computerspiel geeignet und muss vorher abgewägt werden, ob es Sinn macht dies zu verwenden. Mocap-Systeme werden schon seit Jahren in Animationsfilmen verwendet. Jedoch erst seit kurzem in Computergames. Der Grund warum bisher in Computerspielen nicht mit Mocap-Systemen gearbeitet wurde liegt definitiv an den Kosten die dabei entstehen (mehr Personalaufwand, Nacharbeitung und Feinschliff am Computer). Doch mittlerweile ist die Computerspielbranche explodiert und ist ein Multi-Millionen-Dollar Unternehmen geworden, vergleichsweise mit Hollywood. \\
In den letzten Jahren ist in der Computerspielwelt eine neue Genre entstanden. Diese Spiele sind vergleichbar mit einem interaktiven Spielfilm. Der Protagonist trifft Entscheidungen im Spiel, welche wiederum CGI (computer generated imaginery) Video Sequenzen auslösen. Somit wird dem Spieler das Gefühl geboten er selbst wirkt in diesem ``Spielfilm'' mit und taucht in die Geschehnisse des Protagonisten ein. Als Beispiel dient das Computerspiel Heavy Rain.\\
Aber auch andere Gebiete wie die Orthopädie und Sportmedizin haben von Motion Capture profitiert. Beispielsweise wird der Gang eines Menschen aufgezeichnet und kann analysiert werden. Chirurgen verwenden Motion Capturing als eine Art Gestensteuerung um ihre Software zu steuern, da sie ihre Hände aus hygienischen Gründen nicht anders benutzen dürfen. Aber es gibt auch Profigolfer die ihre Schwungtechnik beim Golfen analysieren und versuchen ihre Bewegungen dabei zu verbessern.\\


\section{Quellenangabe}

\begin{itemize}
\item Bildfrequenz - wikipedia.org
\item utah-teekanne - wikipedia.org
\item horse in motion.jpg - wikipedia.org
\item magneticgirl.jpg - http://tyrell-innovations-usa.com
\item optmocap.jpg - lukemccann.files.wordpress.com/2010/09/lolornamocap.jpg
\item Optical Motion Capture: Theory and Implementation - http://smile.uta.edu/Guerra-FilhoRITA05optical.pdf
\item motion capturing - http://animalrace.uni-ulm.de
\item electro mechanic motion capture - http://www.metamotion.com/motion-capture/electro-mechanical-motion-capture.htm
\item Motion Capture to the People: http://www.saidi.se/pdf/moe.pdf
\item kinect.png - http://uatrobotics.blogspot.com/2011/09/led-cube-controller.html
\end{itemize}

%Bildfrequenz, utah-teekanne
%wikipedia.org

%horse in motion.jpg
%wikipedia.org

%magneticgirl.jpg
%http://tyrell-innovations-usa.com

%opt_mocap.jpg
%http://lukemccann.files.wordpress.com/2010/09/lo_lorna_mocap.jpg

%motion capturing
%http://animalrace.uni-ulm.de

