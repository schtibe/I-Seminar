\chapter{Konzepte}
Wir beziehen uns hier auf zwei fundamentale Konzepte, wie man eine Animation erstellen kann. Nämlich das Keyframing und Motion Capturing. 

\section{Keyframing}
Keyframing ist ein Begriff aus der Animationstechnik und bedeutet auf deutsch Schlüsselbildanimation. Diese Technik stammt aus der Produktion von Zeichentrickfilmen. Es werden grobe Bilder (Keyframes) erzeugt, welche die Animation beschreiben. Zwischen den Keyframes werden detailierte Bilder gezeichnet (Interframes) bis sich in der Bewegung ein flüssiges Bild ergibt.

\subsection{Klassisches Keyframing}
\begin{itemize}
\item Regisseur (erstellt Storyboard)
\item Schlüsselzeichner (erstellen Keyframes)
\item Zwischen(phasen)zeichner (erstellen Interframes)
\item Reinzeichner (alle Bilder werden nochmal verbessert und gesäubert)
\end{itemize}

\subsection{Modernes Keyframing}
Die Zwischenzeichner werden nicht mehr benötigt, diese Arbeit wird vom Computer abgenommen. Das zu animierende Objekt steht als 3D-Model in einem Koordinatenkörpers zur Verfügung. Die Interframes werden automatisch mit Rechner erstellt und somit wird die Bildreihenfolge interpoliert. Der Vorteil ist die Senkung des Personalaufwandes. Um die Darstellungen von komplexer Mimik und Gestik zu realisiern,  wird oft Motion Capturing angewandt, weil es zu Zeit- und Kostenaufwändig wäre, dies mit Keyframing zu realiseren. 


\section{Motion Capture}
Motion Capture bedeutet übersetzt nichts anderes als Bewegung aufzeichnen. 
Es wird oft auch MoCap oder MoCa genannt. Motion Capturing wird verwendet 
um Bewegungen in Echtzeit zu erfassen und auf dem Computer darzustellen. 
Später kann man Analysen durchführen oder diese Bewegungen auf ein Model 
projezieren und somit dem Objekt Leben einhauchen. \\
In den letzten Jahren ist diese Methode zu einer der bekanntesten Animationstechniken geworden. 
Durch Filme wie Beispielsweise Star Wars, Herr der Ringe und Avatar aber 
auch durch Computer und Konsolenspiele wie FIFA und Uncharted  
wurde der Bekanntheitsgrad gestärkt.    

\subsection{Menschliche Fortbewegung}

Die Idee ist es eine möglichst realistische Bewegung zu erzeugen, jedoch ist 
dies Beispielsweise mit Keyframing nicht so einfach. Der Gang eines Menschen 
besteht nicht nur aus ''ein Fuss vor den anderen setzen'', sondern eine individuelle 
Folge von verschiedenen Bewegungen des ganzen Körpers. \\
Die künstliche Nachahmung dieser individuellen Bewegungen im Körper gestaltet sich also als schwierig. 
Mit herkömmlichen Animationstechniken ist dies nahezu unmöglich, jedoch beim MoCap 
werden reale Bewegungen eins zu eins übernommen und auf das Model beziehungsweise 
auf den Charakter angewendet. Somit wird die Bewegung nicht künstlich erzeugt sondern 
eine reale Aufzeichnung des Schauspielers. Dadurch wird es zeitsparender und kosteneffizienter
im Vergleich zu anderen Animationstechniken.

\subsection{Motion Sequencing}
Motion Sequencing beschreibt das Aneinanderreihen von Bewegungssequenzen. Erst durch diese Technik
entstehen realistische Animationen. \\
Zum Beispiel: Man hat folgende zwei Bewegungssequenzen, eine
schleichende Person und eine rennende Person. Folglich ist das Motion Sequencing das einander hängen 
der beiden Elemente. Später im Spiel könnte beispielsweise zwischen Schleichen und Sprinten gewechselt
werden. Die Schwierigkeit hierbei liegt, das einander knüpfen aller Elemente ohne einen sichtbaren
Übergang zu erschaffen. Um einen unsichtbaren Übergang zu erhalten, werden mit Hilfe des Rechners 
Zwischenbilder generiert, welche einen weichen Übergang der einzelnen Elemente erzeugen (keyframing).
 	


